\documentclass[12pt, a4paper]{article}
% imports listings
\usepackage{listings, tikz, fancyhdr, enumerate, circuitikz}
\usepackage[shortlabels]{enumitem}
\usepackage{amssymb}
\usetikzlibrary{tikzmark}


% creates a method "insertspace" that seperates text
% by 5mm
\newcommand{\insertspace}{\vspace{5mm}}
\pagestyle{fancy}
\fancyhead[l]{Nodar Aniashvili}
\fancyhead[c]{Computer Architecure - HW 14}
\fancyhead[r]{28th of December}
\fancyfoot[c]{\thepage}
\renewcommand{\headrulewidth}{0.2pt}
\setlength{\headheight}{15pt}

\begin{document}

\noindent 

\begin{enumerate}[start=1,label={\bfseries Question \arabic*:},leftmargin=1in]
    \item 
    \begin{flushleft}
        Find all dependences in the following instruction sequence and draw a dependency graph. Which of them lead to data hazards without forwarding? Which hazards can be resolved by forwarding?
        \newline
        \newline
        \textbf{The Solution:}
    \end{flushleft}
    \begin{lstlisting}
        add $2, $5, $4
        add $4, $2, $5
        sw  $5, 100($2)
        add $3, $2, $4
    \end{lstlisting}
    \begin{tikzpicture}[remember picture, overlay]
        % Circles for register 2
        \draw[red!60!black] (3.21,2.68) circle (2.9mm);
        \draw[red!60!black] (4.2,2.1) circle (2.9mm);

        \coordinate (A) at (3.21,2.68); 
        \coordinate (B) at (4.2,2.1);   
        \path (A) ++(0 : 2.9mm) coordinate (Aedge); 
        \path (B) ++(135 : 2.9mm) coordinate (Bedge);
    
        \draw[-latex, red!60!black] (Aedge) -- (Bedge); 
        
        % Circles for register 4
        \draw[blue!60!black] (3.21,2.1) circle (2.9mm);
        \draw[blue!60!black] (5.2, 2.66) circle (2.9mm);

        \coordinate (C) at (3.21, 2.1);
        \coordinate (D) at (5.2, 2.66);
        \path (C) ++(0 : 2.9mm) coordinate (Cedge);
        \path (D) ++(180 : 2.9mm) coordinate (Dedge);

        \draw[-latex, blue!60!black] (Cedge) -- (Dedge);

        % Circles for register 2
        \draw[blue!60!black] (4.2, 1.1) circle (2.9mm);
        \draw[blue!60!black] (5.2, 1.6) circle (2.9mm);

        \coordinate (G) at (4.2, 1.1);
        \coordinate (H) at (5.2, 1.6);
        \path (G) ++(45 : 2.9mm) coordinate (Gedge);
        \path (H) ++(210 : 2.9mm) coordinate (Hedge);

        \draw[-latex, blue!60!black] (Gedge) -- (Hedge);
    \end{tikzpicture}
 
    \begin{flushleft}  
        The arrows which point upwards signify \textit{anti-dependency}
        \newline
        The arrows which point downwards signify \textit{data dependency}
        \newline
        \newline
        The dependency between s2 registers on the first and second lines leads to read after write hazard, as the 2nd operation may read the s2 register before the 1st one writes to it.
        \newline
        The other 2 dependencies lead to write after read hazard.
        \newline
        All of them can be solved via forwarding.
    \end{flushleft}

    \newpage
    \item 
    \begin{flushleft}
        Consider the following sequence of instructions:
    \end{flushleft}
    \begin{lstlisting}
        add $7, $6, $5
        lw  $6, 100($7)
        sub $7, $6, $81
    \end{lstlisting}
    \begin{flushleft}
        How many clock cycles will it take to execute this code? Draw a diagram t o show  how the code will actually be executed incorporating any stalls. Indicate where operands are forwarded in order to preserve pipeline stalls.
        \newline
        \newline
        \textbf{The Solution:}
    \end{flushleft}
\end{enumerate}
\begin{tabular}{l}
    \textbf{IF}$\rightarrow$\textbf{ID}$\rightarrow$\textbf{Ex}$\rightarrow$\textbf{MEM}$\rightarrow$\textbf{WB} \\
    \hspace{9mm}\textbf{IF}$\rightarrow$\textit{stall}$\rightarrow$\textit{stall}$\rightarrow$\textit{stall}$\rightarrow$\textbf{ID}$\rightarrow$\textbf{Ex}$\rightarrow$\textbf{MEM}$\rightarrow$\textbf{WB} \\
    \hspace{1.9cm}\textbf{IF}$\rightarrow$\textit{stall}$\rightarrow$\textit{stall}$\rightarrow$\textit{stall}$\rightarrow$\textit{stall}$\rightarrow$\textit{stall}$\rightarrow$\textbf{ID}$\rightarrow$\textbf{Ex}$\rightarrow$\textbf{MEM}$\rightarrow$\textbf{WB}
\end{tabular}
\begin{tikzpicture}[remember picture, overlay]
    \coordinate (L) at (-3.5, -0.5);
    \coordinate (R) at (-6.9, 0.3);

    \path (R) ++(-35 : 3mm) coordinate (Redge);
    \path (L) ++(90 : 3.5mm) coordinate (Ledge);

    \draw[-latex, red!60!black] (Redge) -- (Ledge);

    \coordinate (P) at (-11.5, 0.8);

    \path (P) ++(-35 : 3.5mm) coordinate (Pedge);
    \path (R) ++(170 : 3mm) coordinate (Redge2);

    \draw[-latex, red!60!black] (Pedge) -- (Redge2);
\end{tikzpicture}
\begin{center}
    \begin{tabular}{l}
        \hspace{1.7cm}It will take 13 clock cycles to complete this instruction.\\
        \hspace{1.7cm}The red arrows denote forwarding
    \end{tabular}
\end{center}
    
\end{document}